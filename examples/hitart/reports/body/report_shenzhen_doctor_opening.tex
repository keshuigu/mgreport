% !Mode:: "TeX:UTF-8"
%%%%%%%%%%%%%%%%%%%%%%%%%%%%%%%%%%%%%%%%%%%%%%%%%%%%%%%%%%%%%%%%%%%%%%%%%%%%%%%
%                          _
%  _____ ____ _ _ __  _ __| |___ ___
% / -_) \ / _` | '  \| '_ \ / -_|_-<
% \___/_\_\__,_|_|_|_| .__/_\___/__/
%                    |_|
%  _              _         _   _
% | |__ _  _   __| |_  _ __| |_(_)_ _  __ _  _ ___
% | '_ \ || | / _` | || (_-<  _| | ' \/ _| || (_-<
% |_.__/\_, | \__,_|\_,_/__/\__|_|_||_\__|\_, /__/
%       |__/                              |__/
%%%%%%%%%%%%%%%%%%%%%%%%%%%%%%%%%%%%%%%%%%%%%%%%%%%%%%%%%%%%%%%%%%%%%%%%%%%%%%%
\section{课题来源及研究的背景和意义}
\subsection{课题来源}
本研究课题来源项目:人工智能安全模型及评估方法研究
\par
项目编号:MH20200818
\subsection{研究的背景及意义}
近年来,随着科技的快速发展,处理器变得越来越强大,存储器变得越来越便宜,针对各种应用程序的大型图像数据库的部署已经成为现实。由于互联网上的图像信息迅速增长。图像检索技术在各个领域得到了广泛的应用。
基于内容的图像检索(Content Based Image Retrieval,CBIR)\cite{2015Content}是基于颜色、纹理和形状等视觉特征的图像检索。存储在数据库中的每个图像都被提取其特征并与查询图像的特征进行比较。它涉及两个步骤:(1)提取可区分程度的图像特征;(2)匹配这些特征以产生视觉相似的结果。
基于内容的图像检索应用方向也十分多样:(1)安全检查:利用指纹或视网膜扫描等生物信息以获取访问权限(2)知识产权:商标图像注册,将新的候选标记与现有标记进行比较,以确保没有混淆财产所有权的风险。(3)医疗诊断:在医学图像的医学数据库中使用基于内容的图像检索技术,识别类似的过去病例来辅助诊断。
基于内容的图像检索技术依赖于从图像中提取局部的特征信息。尺度不变特征转换(Scale invariant feature transform,SIFT)\cite{loweDistinctiveImageFeatures2004}是用于提取图像局部特征的一种比较流行的方法。此算法由 David Lowe 在1999年所发表,2004年完善总结。
SIFT算法在图像空间中寻找极值点,并提取出其位置、尺度、旋转不变数。SIFT方法获取图像并将其转换为大量局部特征载体。这些特征载体中的每一个都不受图像的任何缩放、旋转或平移的影响。
SIFT的优点在于在很大范围内对仿射失真、噪声的增加和照明的变化都具有很强的鲁棒性。并且SIFT的计算效率很高,可以在标准PC硬件上以近乎实时的性能从典型图像中提取数千个关键点。
\par
由于SIFT的广泛使用,与SIFT相关的隐私和安全问题也引起了高度关注。\cite{9762698}\cite{Qin2014TowardsEP}
SIFT作为一种源自图像的局部特征,包含丰富的图像内容信息。\cite{10214250}事实证明,攻击者可以根据SIFT获取隐私信息。\cite{10.1145/3386082}。另外一篇具有代表性的工作\cite{5995616}表明,可以从一个图像的局部描述符来重建图像,重建后的图像能够表现出人类可理解的内容。
因此,通过利用SIFT进行重建图像的攻击具备了可行性。
图1简要说明了SIFT可能导致的图像内容泄露过程。为了实现图像检索服务,用户将待查询图像的SIFT传输到远程服务提供商。远程服务提供商使用用户上传的SIFT进行基于内容的图像检索,最终将检索到的图像返回给用户,从而完成一次图像检索。
假设攻击者可以在这个过程中获取到用户上传的SIFT,那么,攻击者可以用该SIFT作为输入,利用自己的攻击模型生成与原始图像视觉效果近似的图像。\cite{10.1145/3599589.3599596}\cite{SUN2020102642}
\par
本课题主要关注SIFT反向攻击问题。针对目前已有的反向重建图像模型存在的各个问题,从高效率的反向攻击方法、高精准度的反向攻击方法的角度分别进行研究。
%%%%%%%%%%%%%%%%%%%%%%%%%%%%%%%%%%%%%%%%%%%%%%%%%%%%%%%%%%%%%%%%%%%%%%%%%%%%%%%
\section{国内外在该方向的研究现状及分析}
本章将从SIFT反向攻击研究现状、图像生成模型研究现状描述当前的国内外研究现状并简析。
\subsection{SIFT反向攻击研究现状}
在针对SIFT反向攻击的研究方面,早期的研究工作是由Weinzaepfel等人\cite{5995616}首先证明了从SFT特征恢复图像的可行性。他们建立了保存图像补丁与特征点的外部数据库,使用数据库中数据来拼接图像补丁,从而完成重建原始图像的任务。
Angelo等人\cite{6460288}提出了一种反向优化框架,该框架能够仅依赖特征描述符携带的信息来恢复图像,他们的结果表明导致最佳重建的描述符也会导致最佳检索结果。
Vondrick等人\cite{Vondrick_2013_ICCV}提出了一种基于词典学习的方法来可视化HOG(Histogram of Oriented Gradients)描述符,该方法在各种不同的本地特征之间表现出了高度的可移植性。
Desolneux等人\cite{10.1007/978-3-319-58771-4_11}提出了两种基于Poisson编辑和多尺度方向场组合的重建模型。这些模型能够恢复图像的全局形状和许多几何细节,但无需使用任何外部数据库。
Katod等人\cite{Kato_2014_CVPR}表明,可以从词袋(BoVW)表示中的稀疏局部描述符恢复一些原始图像结构。他们使用大规模的图像数据库来估计局部描述符的空间排列。重建任务转换为了一个关于视觉单词的邻接和全局定位成本的拼图问题。
\par
随着深度卷积神经网络的普及,许多基于深度学习的SIFT反向攻击方法被提出。
Mahendrand等人\cite{Mahendran_2015_CVPR}提出了一种基于神经网络的重建图像通用框架,显著提高了重建图像的效果。
Dosovitski等人\cite{Dosovitskiy_2016_CVPR}的工作将各类特征提取技术视为编码器,利用CNN神经网络设计对应的解码器。结果表明除了对SIFT,HOG,LBP这类浅层特征提取技术重建图像有效外,该模型也可以从卷积神经网络的深层特征出发重建图像。
Pittaluga等人\cite{Pittaluga_2019_CVPR}训练了一个具有U-Net结构的级联网络,从局部特征中揭示场景。该网络有效地处理高度稀疏和不规则的二维点分布以及具有缺失点属性的输入。
Wu等人\cite{9393327}通过使用GANs体系结构作为模型主干,同时利用局部二值模式(LBP)特征来弥补SIFT特征在表示图像空间结构方面的局限性,提出了一种深度生成模型SLI,提高了重建图像的效果。该模型由两个网络组成:一个是LBP重建网络,其目的是将SIFT变换为LBP特征;另一个是图像重建网络,它以变换后的LBP为指导产生重建结果。
Pittaluga等人\cite{Pittaluga_2023_ICCV}最近的一项工作提出了基于数据库的反转攻击和基于聚类的反转攻击,结果表明了即便对SIFT等特征进行后处理,仍然保留了恢复原始图像内容的可行性。
Li等人\cite{liDeepReverseAttack2024}基于GAN模型,构建了一个两阶段的条件引导生成模型,他们重新设计了生成模型的损失函数,加入了对SIFT特征的衡量。重建图像的过程分为两步,首先利用SIFT生成大致的图像,再把该图像和SIFT特征共同作为二阶段模型的输入,使用多尺度融合技术增强图像细节。结果表明了重建图像的效果更优。
综上所述,目前效果较好的SIFT反向攻击方法为基于生成式模型所进行反向攻击,因此,有必要对生成模型进行进一步的研究。
\subsection{图像生成模型研究现状}
目前有三个主流的图像生成模型研究方向,分别是基于似然的模型,生成对抗网络(GAN)以及基于能量的模型。\cite{luoUnderstandingDiffusionModels2022}
基于似然的模型,主要目标是学习为观察到的数据样本分配高似然的模型。代表的模型有自回归模型、流模型和变分自动编码器(VAE)。
生成对抗网络模型中一般包括判别器和生成器共同运行,其中生成器根据隐空间采样数据生成一个图像,判别器则用于区分生成的图像与原始的图像,在生成器和判别器的相互对抗过程中,生成器逐渐靠近原始图像分布。
基于能量的模型又称扩散模型,扩散模型一般由前向扩散过程和反向生成过程组成。其中前向扩散过程将图像逐步添加噪声直至变成随机噪声,反向生成过程则将随机噪声逐步去除噪声直至生成图像数据。
2013年,Kingma[39] 等人提出了变分自编码器(VariationalAuto-Encoder, VAE)模型,通过变分贝叶斯方法,将对原始图像数据的负对数似然的建模优化转为变分下界的计算。编码器将原始图像映射到隐变量,解码器从采样的隐变量恢复原始图像。
2015年,Sohn[40] 等提出了条件变分自编码器(ConditionalVariationalAuto-Encoder, CVAE) 模型,其在VAE的基础上,引入条件概率,使得在生成时能够按照标签条件生成。可以认为,VAE与CVAE的区别在于数据产生方式,VAE是从隐变量采样后使用网络生成图像数据;而CVAE使用标签采样隐变量,在用隐变量与标签使用网络生成图像数据。CVAE在VAE基础上,利用了标签信息,并能够条件生成图像数据。
2017年,Van[41]等提出了向量量化变分自编码器(VectorQuantisation Variational Auto-Encoder, VQ-VAE),其在 VAE 基础上,采用了离散的隐变量,并单独训练一个自回归模型来学习隐变量的先验分布。相比于原始的VAE,VQ-VAE采用了离散编码,并且用了两阶段来生成,让隐变量的先验分布从高斯分布变成可学习的分布,提升了模型的学习能力。
生成对抗模型方面,2014年,Goodfellow[46] 等提出生成对抗网络(Generative Adversarial Networks, GAN) 模型,其通过一个生成器G和一个判别器D的双方博弈完成训练。对于判别器而言,其优化期望能区分输入图像是生成图像的概率;对于生成器而言,其优化期望是能生成判别器难以分辨真伪的图像。
2017年,Arjovsky[49] 等提出WGAN(WassersteinGAN,WGAN)模型,该文献认为原始GAN模型的损失函数中使用的对称的JS散度(Jensen–Shannon Divergence) 不能很好体现两个分布之间的差距,使得在初始阶段分布差距过大时难以训练,KL散度对生成器训练阶段的多样性与真实性的惩罚贡献不均衡,使得模式爆炸而难以生成多样性的样本。使用了Wasserstein距离代替了JS散度,解决训练稳定性问题。
2019年,Karras[51] 等提出了基于风格的StyleGAN网络,该文献使用风格映射网络,将隐空间向量解耦,映射到风格向量上,然后再使用生成器生成图像数据。该文献指出,隐变量的随机向量分布可能和图像特征分布不一致,所以需要风格映射网络,将其映射到图像特征分布,再由图像特征分布生成图像数据。2020年,Karras[52]等在StyleGAN基础上,提出了StyleGAN2模型,针对StyleGAN生成图像中的部分不真实情况,改进了Adain中的归一化问题,优化可损失函数中正则项,添加了路径长度归一化方案,改进了在部分数据集上的生成效果。2021年,Karras[53]等提出了StyleGAN3模型,该文献认为以往的GAN生成过程,特征与像素坐标相关性较高,这导致
生成图像细节难以移动,于是修改了生成器中的非理想上采样滤波器等不合理源,解决了图像坐标与特征粘连问题,实现了图像旋转平移不变形,提升了图像生成的质量。StyleGAN系列,其使用了隐变量和特征的解耦思路,能够将隐空间映射到特征空间,并使其特征空间平滑。但其训练仍有一定难度,计算资源消耗巨大,且有一定的不稳定性。
扩散模型系列,2020 年,Ho[54] 等提出了第一个正式的去噪扩散模型DDPM(Denoising Diffusion Probabilistic Models, DDPM),其包含一个前向的扩散过程和一个反向的生成过程。前向扩散过程中,将原始图像数按马尔科夫过程据逐步添加随机噪声,并逐步变成纯随机噪声;在反向生成过程中,将噪声数据每次去噪并采样,逐步恢复原始数据。DDPM对整个扩散生成过程建模,经过优化将问题转变为预测每一步的随机噪声,并采用神经网络对噪声预测拟合。
2019年,Song[56]等提出了NCSN(Noise-Conditional Score Networks, NCSN),其主要思路为分数匹配方法来估算数据分布的分数函数,并通过朗之万动力学采样实现采样生成。由于数据位于高维空间中的低维流行上,难以估计分数函数,则该文献提出了使用不同程度的噪声对其扰动,并联合估计分数函数。该文献为雏形的扩散模型,且该文献从随机微分方程角度简单探讨了扩散模型的概念,也为了后文出现的基于随机微分方程的扩散模型提供了基础。2020年,Song[57] 等针对扩散模型,提出了ScoreSDE框架统一并且解释了扩散模型。该文献从分数匹配与能量模型角度,提出了基于随机微分方程(StochasticDifferential Equations, SDE) 的去噪分数匹配模型。不同于DDPM的离散形式,使用随机微分方程建模的ScoreSDE是连续形式,正向过程通过SDE求解来注入噪声,将图像数据分布转换到已知的先验分布,并使用神经网络模型学习分数,反向过程通过预测并修正的采样方案,最终将噪声去除并从先验分布转换到数据分布。该文献不仅提出模型,并且将以往的DDPM模型和SMLD模型都统一使用SDE模型表达,实现了对扩散模型的解释与统一。
2022年,Rombach[64] 等提出了隐扩散模型 LDM(Latent Diffusion Models, LDM),因以往的扩散模型直接在图像空间扩散与训练,对计算资源、运算时间消耗大,LDM在隐空间作扩散训练,通过预训练的自编码模型来实现对图像像素空间与隐空间的转换。其中还内嵌条件生成机制,可以在模型中引入多种形态的条件机制,如文本、标签、语音、图像等条件信息。经过实验,其在图像生成、超分辨率、图像修复等诸多下游任务都有很好的表现,后续开源的文生图大模型StableDiffusion就使用的该隐扩散模型架构。
% TODO 多模态
\subsection{国内外文献综述的简析}
目前已有的针对SIFT反向攻击方法中,传统方法效果普遍较差,且部分方法依赖于需要耗费大量精力准备的数据库。基于深度学习的方法虽然取得了优于传统方法的效果,但它们在充分揭示SIFT特征中包含的信息方面仍然存在局限性,并且重建图像的质量还没有达到令人满意的程度。
\par
综合本节内容,目前仍然需要对针对SIFT反向攻击方法的作进一步研究,以提升图像重建的效果和效率。
%%%%%%%%%%%%%%%%%%%%%%%%%%%%%%%%%%%%%%%%%%%%%%%%%%%%%%%%%%%%%%%%%%%%%%%%%%%%%%%
\section{主要研究内容及研究方案}
\subsection{主要研究内容}
(撰写宜使用将来时态,不能只列出论文目录来代替对研究内容的分析论述)
本节主要介绍本课题的两个主要研究内容,包括高精准度的SIFT反向攻击方法研究、高效率的SIFT反向攻击方法研究。
\subsubsection{高精准度的SIFT反向攻击方法研究}
\subsubsection{高效率的SIFT反向攻击方法研究}
\subsection{研究方案}
\subsubsection{SIFT及反向攻击研究}
SIFT原理
SIFT反向攻击通用思路
\subsubsection{基于目标引导扩散模型的SIFT反向攻击研究}
目标引导扩散模型原理
SIFT作为目标,嵌入其中
\subsubsection{基于多模态模型微调的SIFT反向攻击研究}
多模态模型原理,文本和图像如何嵌入同一特征空间
微调手段,训练多模态模型理解SIFT的能力
%%%%%%%%%%%%%%%%%%%%%%%%%%%%%%%%%%%%%%%%%%%%%%%%%%%%%%%%%%%%%%%%%%%%%%%%%%%%%%%
\section{预期达到的目标}
预期目标可分为两个阶段:
(1)完成条件引导DDPM模型构建以及训练过程,使其能够接受未知图像的SIFT,从SIFT出发反向出尽可能与原图像相似的图像。
(2)使用微调方法对当前已有的预训练多模态模型进行训练,使其具备理解SIFT的能力,并能够从SIFT出发尽可能反向原图像。


\section{已完成的研究工作与进度安排}
\subsection{已完成的研究工作和取得的研究成果}
%TODO CFGAN原理
目前,已有相当多针对SIFT进行反向攻击的工作,其中效果较好的是CFGAN。本文深入研究了CFGAN网络结构与实现原理,利用该文章给出的模型,测试针对SIFT的反向攻击效果,得到以下结果。
\subsection{进度安排}
      2024.06~2024.08\quad 查阅相关文献,确定课题内容及方案。
\par  2024.08-2024.10\quad 完成3.2节研究方案中第一步,对现有SIFT反向攻击算法进行研究与分析。
\par  2024.10-2025.01\quad 完成3.2节研究方案中第二步,探索分析现有DDPM模型及算法,以及应用在SIFT反向攻击的可能性,并将适用的算法尝试应用其中。
\par  2025.01-2025.04\quad 继续3.2节研究方案中第二步,对应用于SIFT反向攻击的DDPM模型及算法进行实验,与已有算法进行对比分析。
\par  2025.04-2025.07\quad 完成3.2节研究方案中第三步,设计基于多模态模型的微调方法,利用多模态模型加速SIFT反向攻击实现。
\par  2025.07-2025.10\quad 撰写毕业论文,准备答辩。
%%%%%%%%%%%%%%%%%%%%%%%%%%%%%%%%%%%%%%%%%%%%%%%%%%%%%%%%%%%%%%%%%%%%%%%%%%%%%%%
\section{为完成课题已具备和所需的条件和经费}
为完成本课题,需要如下条件:(1)可供实验使用并带有高性能显卡的计算服务器;(2)Pytorch、Pycharm等软件开发平台与工具;目前以上条件均满足。信息对抗研究所对于本课题给予了充分支持,经费充足。且实验室具有良好的学术氛围与研讨环境,充分支持本课题研究。
%%%%%%%%%%%%%%%%%%%%%%%%%%%%%%%%%%%%%%%%%%%%%%%%%%%%%%%%%%%%%%%%%%%%%%%%%%%%%%%
\section{预计研究过程中可能遇到的困难、问题,以及解决的措施}
课题可能遇到的困难:
\par (1) 在大规模数据集或大模型条件下的实验可能对算力要求较高,实验进度缓、周期长;
\par (2) 对理论到实践的过程掌握不够,理论实践困难;
\par 解决途径
\par (1)通过使用高性能计算设备、高性能显卡并行加速计算速度,使用高效的辅助工具实现类库提升计算效率;
\par (2)加强学习先前工作的实践过程,学习他人从理论到实际的转换方法。
%%%%%%%%%%%%%%%%%%%%%%%%%%%%%%%%%%%%%%%%%%%%%%%%%%%%%%%%%%%%%%%%%%%%%%%%%%%%%%%
\section{主要参考文献}
\bibliographystyle{hithesis}
\bibliography{reference}

% Local Variables:
% TeX-master: "../report"
% TeX-engine: xetex
% End: