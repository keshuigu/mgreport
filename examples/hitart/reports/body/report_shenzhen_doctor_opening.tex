% !Mode:: "TeX:UTF-8"
%%%%%%%%%%%%%%%%%%%%%%%%%%%%%%%%%%%%%%%%%%%%%%%%%%%%%%%%%%%%%%%%%%%%%%%%%%%%%%%
%                          _
%  _____ ____ _ _ __  _ __| |___ ___
% / -_) \ / _` | '  \| '_ \ / -_|_-<
% \___/_\_\__,_|_|_|_| .__/_\___/__/
%                    |_|
%  _              _         _   _
% | |__ _  _   __| |_  _ __| |_(_)_ _  __ _  _ ___
% | '_ \ || | / _` | || (_-<  _| | ' \/ _| || (_-<
% |_.__/\_, | \__,_|\_,_/__/\__|_|_||_\__|\_, /__/
%       |__/                              |__/
%%%%%%%%%%%%%%%%%%%%%%%%%%%%%%%%%%%%%%%%%%%%%%%%%%%%%%%%%%%%%%%%%%%%%%%%%%%%%%%
\section{课题来源及研究的背景和意义}
\subsection{课题来源}
% TODO
\subsection{研究的背景及意义}
近年来,随着科技的快速发展,互联网上的图像信息迅速增长。图像检索技术在各个领域得到了广泛的应用。相当多的图像检索技术依赖于从图像中提取局部的特征信息。
尺度不变特征转换(Scale-invariant feature transform,SIFT)
\cite{loweDistinctiveImageFeatures2004}
是计算机视觉领域中一种比较流行的方法。此算法由 David Lowe 在1999年所发表,2004年完善总结。
SIFT算法在空间尺度中寻找极值点,并提取出其位置、尺度、旋转不变数。SITF的一部分优点在于在图像经历变换或旋转时仍能够表现出出色的匹配性能。此外,SFT对光线变化和噪音也具有具有很强的鲁棒性。
\par
由于SIFT的广泛使用,与SIFT相关的隐私和安全问题也引起了高度关注。
SIFT特征作为一种源自图像的局部特征,包含丰富的图像内容信息。事实证明,反向攻击可以根据SIFT特征重建原始图像。
图1简要说明了SIFT功能导致的图像内容泄露的过程。为了实现图像检索服务,本地设备的计算能力限制要求用户将查询的SFT特征传输到远程服务提供商。
因此,与远程服务提供商共享的图像特征可能会用于通过反向恢复攻击重建原始图像。

\section{国内外在该方向的研究现状及分析}
本章将从SIFT特征恢复攻击研究现状、图像生成模型研究现状描述当前的国内外研究现状并简析。
\subsection{SIFT特征恢复攻击研究现状}
\subsection{图像生成模型研究现状}
\subsection{国内外文献综述的简析}
(综合评述:国内外研究取得的成果,存在的不足或有待深入研究的问题)
为了揭示局部特征内的信息并评估滥用局部特征造成的潜在隐私风险,已经提出了许多图像恢复方法来从局部特征恢复图像。开创性工作表明,在一系列条件和配置下可以实现对特征描述符的反向攻击。此外,许多不同的方法已经表明,反向攻击可以应用于广泛的传统图像特征,包括SIFT、定向梯度图(HOG)和词袋(BoW)。上述方法已经证明了对局部特征进行反向攻击的可能性。但重建结果与原始图像之间存在显着差异,确定SIFT特征造成的安全风险仍然具有挑战性。
\section{主要研究内容及研究方案}
\subsection{主要研究内容}
(撰写宜使用将来时态,不能只列出论文目录来代替对研究内容的分析论述)
\subsection{研究方案}
\section{预期达到的目标}
\section{已完成的研究工作与进度安排}
\subsection{已完成的研究工作和取得的研究成果}
应用SIFT特征提取技术,预先处理CelebA等公开数据集
\par
调研CFGAN网络结构,使用该模型测试针对SIFT特征的恢复攻击效果
\subsection{进度安排}
\begin{table}[h]
  \centering
  \caption{进度安排}\label{table3}
  \begin{tabularx}{0.8\textwidth}{lX}
    \toprule
    2024.06~2024.08 & 查阅相关文献,确定课题内容及方案。\\
    2024.08-2024.10 & 完成3.2节研究方案中第一步,对现有SIFT特征恢复攻击算法进行研究与分析。\\
    2024.10-2025.01	& 完成3.2节研究方案中第二步,探索分析现有DDPM模型及算法,以及应用在SIFT特征恢复攻击的可能性,并将适用的算法尝试应用其中。\\
    2025.01-2025.04 & 继续3.2节研究方案中第二步,对应用于SIFT特征恢复攻击的DDPM模型及算法进行实验,与已有算法进行对比分析。\\
    2025.04-2025.07 & 完成3.2节研究方案中第三步,设计基于多模态模型的微调方法,利用多模态模型加速SIFT特征恢复攻击实现。\\
    2025.07-2025.10 & 撰写毕业论文,准备答辩。\\
    \bottomrule
    \end{tabularx}
\end{table}
\section{为完成课题已具备和所需的条件和经费}
\section{预计研究过程中可能遇到的困难、问题,以及解决的措施}
\section{主要参考文献}
\bibliographystyle{hithesis}
\bibliography{reference}

% Local Variables:
% TeX-master: "../report"
% TeX-engine: xetex
% End: