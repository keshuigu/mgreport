% !Mode:: "TeX:UTF-8"
%%%%%%%%%%%%%%%%%%%%%%%%%%%%%%%%%%%%%%%%%%%%%%%%%%%%%%%%%%%%%%%%%%%%%%%%%%%%%%%
%                          _
%  _____ ____ _ _ __  _ __| |___ ___
% / -_) \ / _` | '  \| '_ \ / -_|_-<
% \___/_\_\__,_|_|_|_| .__/_\___/__/
%                    |_|
%  _              _         _   _
% | |__ _  _   __| |_  _ __| |_(_)_ _  __ _  _ ___
% | '_ \ || | / _` | || (_-<  _| | ' \/ _| || (_-<
% |_.__/\_, | \__,_|\_,_/__/\__|_|_||_\__|\_, /__/
%       |__/                              |__/
%%%%%%%%%%%%%%%%%%%%%%%%%%%%%%%%%%%%%%%%%%%%%%%%%%%%%%%%%%%%%%%%%%%%%%%%%%%%%%%
\section{课题来源及研究的背景和意义}
\subsection{课题来源}
本研究课题来源项目:人工智能安全模型及评估方法研究
\par
项目编号:MH20200818
\subsection{研究的背景及意义}
近年来,随着科技的快速发展,处理器变得越来越强大,存储器变得越来越便宜,针对各种应用程序的大型图像数据库的部署已经成为现实。由于互联网上的图像信息迅速增长。图像检索技术在各个领域得到了广泛的应用。
基于内容的图像检索(Content Based Image Retrieval,CBIR)\cite{2015Content}是基于颜色、纹理和形状等视觉特征的图像检索。存储在数据库中的每个图像都被提取其特征并与查询图像的特征进行比较。它涉及两个步骤:(1)提取可区分程度的图像特征;(2)匹配这些特征以产生视觉相似的结果。
基于内容的图像检索应用方向也十分多样:(1)安全检查:利用指纹或视网膜扫描等生物信息以获取访问权限(2)知识产权:商标图像注册,将新的候选标记与现有标记进行比较,以确保没有混淆财产所有权的风险。(3)医疗诊断:在医学图像的医学数据库中使用基于内容的图像检索技术,识别类似的过去病例来辅助诊断。
基于内容的图像检索技术依赖于从图像中提取局部的特征信息。尺度不变特征转换(Scale invariant feature transform,SIFT)\cite{loweDistinctiveImageFeatures2004}是用于提取图像局部特征的一种比较流行的方法。此算法由 David Lowe 在1999年所发表,2004年完善总结。
SIFT算法在图像空间中寻找极值点,并提取出其位置、尺度、旋转不变数。SIFT方法获取图像并将其转换为大量局部特征载体。这些特征载体中的每一个都不受图像的任何缩放、旋转或平移的影响。
SIFT的优点在于在很大范围内对仿射失真、噪声的增加和照明的变化都具有很强的鲁棒性。并且SIFT的计算效率很高,可以在标准PC硬件上以近乎实时的性能从典型图像中提取数千个关键点。
\par
由于SIFT的广泛使用,与SIFT相关的隐私和安全问题也引起了高度关注。\cite{9762698}\cite{Qin2014TowardsEP}
SIFT作为一种源自图像的局部特征,包含丰富的图像内容信息。\cite{10214250}事实证明,攻击者可以根据SIFT获取隐私信息。\cite{10.1145/3386082}。另外一篇具有代表性的工作\cite{5995616}表明,可以从一个图像的局部描述符来重建图像,重建后的图像能够表现出人类可理解的内容。
因此,通过利用SIFT进行重建图像的攻击具备了可行性。
图1简要说明了SIFT可能导致的图像内容泄露过程。为了实现图像检索服务,用户将待查询图像的SIFT传输到远程服务提供商。远程服务提供商使用用户上传的SIFT进行基于内容的图像检索,最终将检索到的图像返回给用户,从而完成一次图像检索。
假设攻击者可以在这个过程中获取到用户上传的SIFT,那么,攻击者可以用该SIFT作为输入,利用自己的攻击模型生成与原始图像视觉效果近似的图像。\cite{10.1145/3599589.3599596}\cite{SUN2020102642}
\par
本课题主要关注SIFT反向攻击问题。针对目前已有的反向重建图像模型存在的各个问题,从高效率的反向攻击方法、高精准度的反向攻击方法的角度分别进行研究。
\section{国内外在该方向的研究现状及分析}
本章将从SIFT反向攻击研究现状、图像生成模型研究现状描述当前的国内外研究现状并简析。
\subsection{SIFT反向攻击研究现状}
在针对SIFT反向攻击的研究方面,早期的研究工作并未采用机器学习的方式。
Weinzaepfel等人\cite{5995616}首先证明了从SFT特征恢复图像的可行性。他们建立了保存图像补丁与特征点的外部数据库,使用数据库中数据来拼接图像补丁,从而完成重建原始图像的任务。
Angelo等人[15]提出了一种反向优化框架,该框架能够仅依赖特征描述符携带的信息来恢复图像。
Vondrick等人[16]提出了一种基于词典学习的方法来可视化HOG描述符,该方法在各种不同的本地特征之间表现出了高度的可移植性。
Kato和Harada [18]表明,可以从词袋(BoW)表示中的稀疏局部描述符恢复一些原始图像结构。
\par
随着深度卷积神经网络的普及,许多基于深度学习的反向攻击方法被提出[19]、[20]、[21]、[22]、[23]。
Mahendrand和Vedaldi[19]提出了一种基于神经网络的图像恢复通用框架,显著提高了图像恢复的性能。
Dosovitski和Brox[20]提出的模型采用编解码器结构来根据局部特征重建图像。此外,该模型已成功地应用于卷积神经网络的高级特征。
Pittaluga等人。[21]训练了一个具有U-Net结构的级联网络,从局部特征中揭示场景。该网络有效地处理高度稀疏和不规则的二维点分布以及具有缺失点属性的输入。
吴和周[22]通过使用Gans体系结构作为模型主干提高了图像恢复的性能。此外,利用局部二值模式(LBP)特征来弥补SIFT特征在表示图像空间结构方面的局限性。
Pittaluga和Zang[23]最近的一项工作提出了两种新的反转攻击,表明了使用来自后处理的图像特征恢复原始图像内容的可行性
李等人基于GAN模型,构建了一个两阶段的条件引导生成模型,首先利用SIFT生成大致的图像,再将此图像与SIFT以一种巧妙的方式融合,共同作为二阶段模型的输入,获得了比较好的重建图像的效果。
综上所述,目前效果较好的SIFT反向攻击方法为基于生成式模型所进行反向攻击,因此,有必要对生成模型进行进一步的研究。
\subsection{图像生成模型研究现状}
目前有三个主流的图像生成模型研究方向,分别是生成对抗网络(GAN),基于似然的模型,基于能量的模型。\cite{luoUnderstandingDiffusionModels2022}
生成对抗网络,对复杂分布的采样过程进行建模,该过程是以对抗方式学习的。
另一类生成模型被称为“基于可能性”,旨在学习为观察到的数据样本分配高可能性的模型。这包括自回归模型、规范化流和变分自动编码器(VAE)。
基于能量的模型。其中将分布学习为任意灵活的能量函数,然后进行标准化。从另一个角度,也可以将图像生成模型分为无条件生成模型和条件引导生成模型两个类别。
\subsection{国内外文献综述的简析}
(综合评述:国内外研究取得的成果,存在的不足或有待深入研究的问题)
尽管这些方法取得了优于传统方法的效果,但它们在充分揭示SIFT特征中包含的信息方面仍然存在局限性,并且重建图像的质量不令人满意。
\par 为了揭示局部特征内的信息并评估滥用局部特征造成的潜在隐私风险,已经提出了许多图像反向方法来从局部特征反向图像。开创性工作表明,在一系列条件和配置下可以实现对特征描述符的反向攻击。此外,许多不同的方法已经表明,反向攻击可以应用于广泛的传统图像特征,包括SIFT、定向梯度图(HOG)和词袋(BoW)。上述方法已经证明了对局部特征进行反向攻击的可能性。但重建结果与原始图像之间存在显着差异,确定SIFT造成的安全风险仍然具有挑战性。
\section{主要研究内容及研究方案}
\subsection{主要研究内容}
(撰写宜使用将来时态,不能只列出论文目录来代替对研究内容的分析论述)
本节主要介绍本课题的两个主要研究内容,包括高精准度的SIFT反向攻击方法研究、高效率的SIFT反向攻击方法研究。
\subsubsection{高精准度的SIFT反向攻击方法研究}
\subsubsection{高效率的SIFT反向攻击方法研究}
\subsection{研究方案}
\subsubsection{SIFT及反向攻击研究}
SIFT原理
SIFT反向攻击通用思路
\subsubsection{基于目标引导扩散模型的SIFT反向攻击研究}
目标引导扩散模型原理
SIFT作为目标,嵌入其中
\subsubsection{基于多模态模型微调的SIFT反向攻击研究}
多模态模型原理,文本和图像如何嵌入同一特征空间
微调手段,训练多模态模型理解SIFT的能力
\section{预期达到的目标}
预期目标可分为两个阶段:(1)完成条件引导DDPM模型构建以及训练过程,使其能够接受未知图像的SIFT,从SIFT出发反向出尽可能与原图像相似的图像。
(2)使用微调方法对当前已有的预训练多模态模型进行训练,使其具备理解SIFT的能力,并能够从SIFT出发尽可能反向原图像。
\section{已完成的研究工作与进度安排}
\subsection{已完成的研究工作和取得的研究成果}
%TODO CFGAN原理
目前,已有相当多针对SIFT进行反向攻击的工作,其中效果较好的是CFGAN。本文深入研究了CFGAN网络结构与实现原理,利用该文章给出的模型,测试针对SIFT的反向攻击效果,得到以下结果。
\subsection{进度安排}
\begin{table}[h]
  \centering
  \caption{进度安排}\label{table3}
  \begin{tabularx}{0.8\textwidth}{lX}
    \toprule
    2024.06~2024.08 & 查阅相关文献,确定课题内容及方案。\\
    2024.08-2024.10 & 完成3.2节研究方案中第一步,对现有SIFT反向攻击算法进行研究与分析。\\
    2024.10-2025.01	& 完成3.2节研究方案中第二步,探索分析现有DDPM模型及算法,以及应用在SIFT反向攻击的可能性,并将适用的算法尝试应用其中。\\
    2025.01-2025.04 & 继续3.2节研究方案中第二步,对应用于SIFT反向攻击的DDPM模型及算法进行实验,与已有算法进行对比分析。\\
    2025.04-2025.07 & 完成3.2节研究方案中第三步,设计基于多模态模型的微调方法,利用多模态模型加速SIFT反向攻击实现。\\
    2025.07-2025.10 & 撰写毕业论文,准备答辩。\\
    \bottomrule
    \end{tabularx}
\end{table}
\section{为完成课题已具备和所需的条件和经费}
为完成本课题,需要如下条件:(1)可供实验使用并带有高性能显卡的计算服务器;(2)Pytorch、Pycharm等软件开发平台与工具;目前以上条件均满足。信息对抗研究所对于本课题给予了充分支持,经费充足。且实验室具有良好的学术氛围与研讨环境,充分支持本课题研究。
\section{预计研究过程中可能遇到的困难、问题,以及解决的措施}
课题可能遇到的困难:
\par (1) 在大规模数据集或大模型条件下的实验可能对算力要求较高,实验进度缓、周期长;
\par (2) 对理论到实践的过程掌握不够,理论实践困难;
\par 解决途径
\par (1)通过使用高性能计算设备、高性能显卡并行加速计算速度,使用高效的辅助工具实现类库提升计算效率;
\par (2)加强学习先前工作的实践过程,学习他人从理论到实际的转换方法。
\section{主要参考文献}
\bibliographystyle{hithesis}
\bibliography{reference}

% Local Variables:
% TeX-master: "../report"
% TeX-engine: xetex
% End: